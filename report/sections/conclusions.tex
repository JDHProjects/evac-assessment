Results clearly show that local sensing is the superior collision sensing method to space aware (search for available space) sensing. This is likely due to space aware sensing being significantly more sensitive and therefore easy for progress to be overwhelmed by random mutations, as evidenced by \verb|space_aware-n+n| being able to successfully learn how to gain an average of approximately 20 score, while \verb|space_aware-standard| struggled to make any progress at all. An added benefit of local sensing performing significantly better compared to space aware relates to time taken to run a generation. For an example 50 generations evolution of a population of 200 individuals, local sensing completed all generation evolutions in 6.2 seconds, while space aware took 51.8 seconds. This 8.4x slowdown in evolution drastically increases training time in later generations.

With our best sensing method of local collision sensing selected, conclusions can be drawn about our chosen evolutionary selection methods. Unfortunately the two datasets (\verb|local-standard| and \verb|local-n+n|) were found to be not statistically significant. While the datasets were not statistically significant, future work should focus on longer generation testing, with the goal of finding if these two datasets diverge in later generations. At the stopping point of 500 generations in this report, both datasets showed no signs of reaching a plateau. Logic suggests that n+n evolution should plateau at a higher score than standard evolution due to its reduced susceptibility to mutations that decrease score, however without data to show this, no conclusion regarding this can be drawn from this report.

Additionally future work may wish to investigate the effects of implementing crossover, or changing mutation probability values on each of the algorithms detailed in this report. All algorithms in this report used the same standard values as control variables, however each algorithm may benefit from tuning these parameters.