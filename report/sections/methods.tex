\subsection{Changes to Provided Code}
While the snake game discussed in this report is functionally the same as the initially supplied code provided with the assessment, some large changes have been made to the structure. Most notably, the snake game and neural network has been split into multiple separate scripts for ease of development. Details on all provided files can be found in appendix \ref{appendix:a}.

\subsection{Chosen Evolutionary Method}
A Genetic Algorithm (GA) was used to evolve weights in a Neural Network (NN). Input nodes to this neural network were the snake sensing functions, with four output nodes representing snake direction: left, right, up and down. Heaton \cite{heaton} discusses that most problems to not require more than one hidden layer, therefore a single hidden layer was used for the neural network with the number of hidden layer nodes corresponding to the sum of the input and output layer nodes. Using a single hidden layer has a benefit of reducing overfitting compared to a multiple hidden layer network, along with decreasing time taken to train the network. A softmax function is used at the output layer to obtain the direction the snake should take, where left=0, right=1, up=2 and down=3. 

\subsection{Development of Different Snakes}
Four differing snakes were developed for the snake game and will be referred to in this report as:
\begin{enumerate}[itemsep=-0.5em]
  \item \verb|local-standard|
  \item \verb|space_aware-standard|
  \item \verb|local-n+n|
  \item \verb|space_aware-n+n|
\end{enumerate}
Two directions of investigation were performed on the snake game, where each name is comprised of these two components separated by a hyphen. The first direction investigated the effect of differing collision sensing functions (\verb|local| against \verb|space-aware|) while the second investigated changing the generation selection method (\verb|standard| against \verb|n+n|).

\subsection{Sensing Functions}
Sensing functions were created for the snake to use to learn about its environment, these can be classified into two groups; food sensing functions and collision sensing functions. All sensing assumes grid format of \verb|[y,x]| and directions are relative to the grid, not the current direction of the snake.

\subsubsection{Food Sensing}
Food sensing was constant throughout all developed snakes. Multiple methods of food sensing were explored during development including passing the exact position of the food, calculating the manhattan distance between food and snake, and sensing the quadrant of the grid the food was in relative to the head of the snake. The chosen method instead simply checks if food is in the direction specified, for left and right the x axis of the snake is checked while up and down checks the right axis. The sensing function returns a boolean value; True if food is present and False if not. 

If the Snake is directly in line with food in that direction, both directions on the axis will return False. For example, a snake head at position [4,4] and food at [6,6] would return True from \verb|downFood| and \verb|rightFood|, and False from \verb|upFood| and \verb|leftFood|. A snake head at position [4,4] and food at [6,4] would return True from \verb|downFood| and False from all other sense directions.

\subsubsection{Collision Sensing}
Two methods of collision sensing were developed, the first method of collision sensing is referred to as \verb|local| and was used by both \verb|local-standard| and \verb|local-n+n|. Each direction has two collision sensing functions that both check based on a given position; one used to sense if the position is within a wall, and the other used to sense if the position is within the snake body. Both of these sensing functions return True if a collision is detected, False otherwise. These positions are given for each direction by calculating the position of the snake head if the snake moved in the given direction. For example, if the snake head is at position [4,4], the position given to \verb|senseWall| and \verb|senseTail| would be [4,3] when \verb|leftWall| and \verb|leftTail| are called. 

These wall and tail functions could be combined into a single function to detect if a collision has occurred, however this was not implemented as providing both collision detection components separately allows a snake to learn separate behaviors depending on collision type, for example chasing it's tail or circling the grid perimeter.

The second method of collision sensing is referred to as \verb|space_aware| and was used by both \verb|space_aware-standard| and \verb|space_aware-n+n|. This method was created to attempt to prevent the snake being trapped with no space to go. To create this sensing function, a recursive search was performed for a given position that counted all available spaces from that position. This search is performed at each of the four possible positions that the snake can move to (up, down, left and right).

\subsection{Population Selection Between Generations}
Two selection methods have been developed for this report. Both generation to generation selection methods use a tournament selection method with a size of 3 to select individuals to evolve. Where the two selection methods standard and n+n differ is their assignment of the population to be used for the next generation. Standard evolution simply uses tournament selection to select a full population of individuals, these individuals are then evaluated and all selected individuals are assigned to the population. 

n+n evolution, as discussed by Yeh et al \cite{nplusn}, uses the concept of preserving best individuals from earlier generations without mutation. This is performed by tournament selecting a full population of individuals that are then evaluated. n+n evolution combines the best half of the previous generation population with the best half of these newly evaluated individuals, to preserve individuals with promising weighting from being overwhelmed by random mutations.

\subsection{Fitness, Mutation and Crossover}
Multiple fitness functions were tested before settling on one fitness function for all snakes that gave best performance. This fitness function was detailed by Haber \cite{craig} and is of the form $(\texttt{score}^3) * \texttt{timeAlive}$, where \texttt{timeAlive} is a counter incremented each move. A starvation function of $\min(5 * \texttt{snakeLength}, 200)$ was implemented alongside this to prevent snakes developing behavior that maximises time alive over score. An additional variable \texttt{timeSinceEaten} was incremented each move and reset to 0 when food was eaten by the snake, this was the variable compared to in the starvation function. If a snake died to starvation, \texttt{timeSinceEaten} was subtracted from \texttt{timeAlive} to discourage future starvation promoting behaviour.

Mutation probability was explored in two differing ways, the first being \texttt{MUTPB} which is the probability that an individual is selected for mutation. The second was \texttt{indpb}, which is the probability that a single attribute of an individual is mutated. Gaussian mutation was chosen to mutate with, and through preliminary testing ideal values were found to be $\texttt{MUTPB} = 1, \texttt{indpb} = 0.1$.

Through preliminary testing, both single and two-point crossover were found to harm the evolution of this implementation of snake AI evolution, and were subsequently removed from final versions of all algorithms by setting the constant \texttt{CXPB} to 0, for a 0 probability of an individual experiencing crossover.